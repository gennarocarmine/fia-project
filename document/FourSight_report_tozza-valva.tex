\documentclass[12pt]{article}

% --- Lingua e Codifica ---
\usepackage[italian,english]{babel}

% --- Matematica ---
\usepackage{amsmath}
\usepackage{mathtools}
\usepackage{amssymb}
\usepackage{bm}

% --- Colori e Grafica ---
\usepackage[table,xcdraw,svgnames]{xcolor} 
\usepackage{graphicx} % Required for inserting images & resizing

% --- Tabelle ---
\usepackage{array}
\usepackage{multirow}
\usepackage{longtable}
\usepackage{tabularx}
\usepackage{booktabs}
\usepackage{adjustbox}

% --- Formattazione e Layout ---
\usepackage[italian]{minitoc}
\usepackage{fancybox}
\usepackage{fancyhdr}
\usepackage{lscape}
\usepackage{placeins}
\usepackage{float}
\usepackage{caption}
\usepackage{soul}

% --- Utilità e Codice ---
\usepackage{verbatim}
\usepackage{url}
\usepackage{listings}
\usepackage{makeidx}
\usepackage{comment}

% --- Bibliografia ---
\usepackage{biblatex}
\addbibresource{sample.bib}

% --- Collegamenti (Caricare per ultimo) ---

%\titleformat{\chapter}{\normalfont\huge}{\textbf\thechapter.}{20pt}%{\huge\textbf}

%inizio documento
\begin{document}
\selectlanguage{italian}

%inizio copertina
\begin{titlepage}
\begin{center}
	\begin{figure}
    	\includegraphics[width=3.0cm, height=3.0cm]{images/unisa.png}
    	\centering
    \end{figure}
	{\Large Università degli Studi di Salerno}\\[0.2truecm]
	{\large Dipartimento di Informatica\\Corso di Laurea Triennale in Informatica}\\
	\hrulefill
	\vfill
	{\large Fondamenti di Intelligenza Artificiale (FIA)}\\[0.2truecm]
    %{\large Project Proposal}\\[0.2truecm]
	%{\Large Informatica}\\
	\vfill
	{\LARGE {\bf FourSight}}
	
	\vfill\vfill
	
	
	{\bf Docente} \hfill {\bf Studenti}\ \hfill  {\bf Matricola}\  \\
	Prof.  Fabio Palomba \hfill Tozza Gennaro Carmine \hfill 0512120382 \\
    \hfill \ \ \ \ \ \ \ \ \ \ \ \ \ \ \ \ Valva Lorenzo \hfill 0512119639 \\
	\vfill
    [\url{https://github.com/gennarocarmine/fia-project.git}]
    \vfill
	\hrulefill 
	\begin{center} Anno Accademico 2025-2026 \end{center}
	
\end{center}
\end{titlepage}
%fine copertina

\section{Introduzione}
Il periodo festivo porta con sé panettoni, calze e gli immancabili parenti che si lamentano dell'Intelligenza Artificiale mentre guardano video di gattini che ballano sui social. Ma porta anche le giocate a carte. È stato proprio durante l'ennesima partita persa a Sette e Mezzo che è sorta la domanda: "E se al nostro posto giocasse un'IA?".\\\\
\noindent
Da questa sfida nasce \textbf{(IA)'M FINE}. Il nome è un gioco di parole che richiama la tipica frase "Sto bene" (I'm fine), utilizzata dal giocatore per indicare che non desidera ricevere ulteriori carte.\\\\
\noindent
Per chi avesse bisogno di un ripasso delle regole ufficiali, sono disponibili al seguente indirizzo: \\
\url{https://www.sisal.it/Store/Skill/Download/setteemezzo-ga.pdf}.

\section{Definizione del problema} 
Forza 4 è un gioco di strategia a turni a somma zero che si svolge su una griglia verticale di 6 righe per 7 colonne. L'obiettivo è allineare quattro gettoni del proprio colore in orizzontale, verticale o diagonale, sfruttando la meccanica della gravità. \\\\
\noindent
Dal punto di vista dell'Intelligenza Artificiale, il gioco rappresenta un problema a informazione perfetta e deterministico. La sfida risiede nella complessità computazionale. L'obiettivo è calcolare l'azione ottimale (scelta della colonna) prevedendo l emosse future dell'avversario per massimizzare le possibilità di vittoria.

\subsection{Obiettivi} 
L'obiettivo principale del progetto è sviluppare e confrontare due diversi approcci per la risoluzione del gioco "Forza 4": uno basato sulla teoria della ricerca nello spazio degli stati e uno basato sull'apprendimento automatico supervisionato. Il sistema non utilizzerà dataset precostruiti, ma genererà autonomamente i dati necesaari per l'addestramento. Nello specifico, gli obiettivi sono:
\begin{itemize}
    \item \textbf{Generare un Dataset:} Creare un dataset di training personalizzato che superi i limiti delle simulazioni puramente casuali (Random vs Random). Il dataset verrà generato registrando partite miste:
     \begin{itemize}
        \item Bot Random vs Bot Random: Per esplorare lo spazio degli stati in modo ampio.
        \item Minimax vs Random: Per insegnare alla rete neurale pattern strategici, mosse di blocco e sequenze vincenti, garantendo dati di qualità superiore per l'addestramento.
     \end{itemize}
     L'obiettivo è ottenere un dataset ricco e variegato che consenta alla rete neurale di apprendere strategie efficaci.
    \item \textbf{Implementare la pipeline di Ricerca(Minimax):} Sviluppare un agente basato sull'algoritmo Minimax con Alpha-Beta Pruning. Questo modulo avrà il duplice scopo di:
     \begin{itemize}
        \item Fornire un avversario algoritmico "forte" in grado di calcolare la mossa ottimale esplorando l'albero di gioco.
        \item Agire come "oracolo" per la generazione dei dati di training.
     \end{itemize}
    \item \textbf{Implementare la pipeline di Apprendimento (MLP):} Sviluppare e addestrare una rete neurale (MLP) utilizzando il dataset generato. L'obiettivo è ottenere un modello in grado di predire la mossa migliore istantaneamente (approssimando la funzione di valutazione), imitando la logica del Minimax ma con tempi di risposta drasticamente inferiori.
    \item \textbf{Sviluppare un'interfaccia interattiva:} Creare un interfaccia grafica che permetta all'utente di sfidare entrambe le IA (Minmax e MLP).
\end{itemize}

\subsection{Specifica PEAS} 
La descrizione formale dell'ambiente operativo dell'agente è definita secondo il modello PEAS (\textit{Performance, Environment, Actuators, Sensors}):

\begin{itemize}
    \item \textbf{Performance (Misure di Prestazione):}
    \begin{itemize}
        \item Efficienza: Numero di mosse minimo per raggiungere la vittoria.
        \item Correttezza: Evitare mosse non valide (es. inserire in una colonna piena).
        \item Velocità di risposta: Tempo impiegato per calcolare la mossa.
        \item Tasso di vittoria: Percentuale di partite vinte contro vari tipi di avversari.
    \end{itemize}
    
    \item \textbf{Environment (Ambiente):}
    \begin{itemize}
        \item Griglia di gioco:Una matrice 6 X 7.
        \item Avversario: Un essere umano o un'altra IA.
    \end{itemize}
    
    \item \textbf{Actuators (Attuatori/Azioni):}
    \begin{itemize}
        \item \textit{Inserimento gettone}: l'attuatore sceglie una delle 7 colonne disponibili.
        \item \textit{Segnalazione}: comunicazione della mossa o dichiarazione di vittoria/resa.
    \end{itemize}
    
    \item \textbf{Sensors (Sensori/Percezioni):}
    \begin{itemize}
        \item Lettore di matrice: Funzione che scansiona lo sttao attuale della scacchiera per sapere dove sono i propri gettoni, quelli dell'avversario e gli spazi vuoti.
        \item Rilevatore di turno: Funziona che indica alla'gente quando è il suo momento di agire.
    \end{itemize}
\end{itemize}

\subsubsection{Caratteristiche dell’ambiente} 
L'ambiente di gioco Forza 4 presenta le seguenti proprietà:
\begin{itemize}
    \item \textbf{Completamente Osservabile:} L'agente vede l'intera griglia di gioco, non ci osno informazioni nascoste.
    \item \textbf{Deterministico:} Lo stato successivo dell'ambiente è completamente determinato dallo stato corrente e dall'azione eseuita dall'agente.
    \item \textbf{Discreto:} L'ambiente fornisce un numero limitato di percezioni e azioni distinte, chiaramente definite.
\end{itemize}

\subsection{Analisi del problema} 

\section{Soluzione del problema}

\section{Riferimenti}

\printbibliography

\end{document}