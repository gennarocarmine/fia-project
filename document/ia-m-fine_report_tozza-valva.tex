\documentclass[12pt]{article}

% --- Lingua e Codifica ---
\usepackage[italian,english]{babel}

% --- Matematica ---
\usepackage{amsmath}
\usepackage{mathtools}
\usepackage{amssymb}
\usepackage{bm}

% --- Colori e Grafica ---
\usepackage[table,xcdraw,svgnames]{xcolor} 
\usepackage{graphicx} % Required for inserting images & resizing

% --- Tabelle ---
\usepackage{array}
\usepackage{multirow}
\usepackage{longtable}
\usepackage{tabularx}
\usepackage{booktabs}
\usepackage{adjustbox}

% --- Formattazione e Layout ---
\usepackage[italian]{minitoc}
\usepackage{fancybox}
\usepackage{fancyhdr}
\usepackage{lscape}
\usepackage{placeins}
\usepackage{float}
\usepackage{caption}
\usepackage{soul}

% --- Utilità e Codice ---
\usepackage{verbatim}
\usepackage{url}
\usepackage{listings}
\usepackage{makeidx}
\usepackage{comment}

% --- Bibliografia ---
\usepackage{biblatex}
\addbibresource{sample.bib}

% --- Collegamenti (Caricare per ultimo) ---

%\titleformat{\chapter}{\normalfont\huge}{\textbf\thechapter.}{20pt}%{\huge\textbf}

%inizio documento
\begin{document}
\selectlanguage{italian}

%inizio copertina
\begin{titlepage}
\begin{center}
	\begin{figure}
    	\includegraphics[width=3.0cm, height=3.0cm]{images/unisa.png}
    	\centering
    \end{figure}
	{\Large Università degli Studi di Salerno}\\[0.2truecm]
	{\large Dipartimento di Informatica\\Corso di Laurea Triennale in Informatica}\\
	\hrulefill
	\vfill
	{\large Fondamenti di Intelligenza Artificiale (FIA)}\\[0.2truecm]
    %{\large Project Proposal}\\[0.2truecm]
	%{\Large Informatica}\\
	\vfill
	{\LARGE {\bf (IA)'M FINE: IA PER IL GIOCO DEL SETTE E MEZZO}}
	
	\vfill\vfill
	
	
	{\bf Docente} \hfill {\bf Studenti}\ \hfill  {\bf Matricola}\  \\
	Prof.  Fabio Palomba \hfill Tozza Gennaro Carmine \hfill 0512120382 \\
    \hfill \ \ \ \ \ \ \ \ \ \ \ \ \ \ \ \ Valva Lorenzo \hfill 0512119639 \\
	\vfill
    [\url{https://github.com/gennarocarmine/fia-project.git}]
    \vfill
	\hrulefill 
	\begin{center} Anno Accademico 2025-2026 \end{center}
	
\end{center}
\end{titlepage}
%fine copertina

\section{Introduzione}
Il periodo festivo porta con sé panettoni, calze e gli immancabili parenti che si lamentano dell'Intelligenza Artificiale mentre guardano video di gattini che ballano sui social. Ma porta anche le giocate a carte. È stato proprio durante l'ennesima partita persa a Sette e Mezzo che è sorta la domanda: "E se al nostro posto giocasse un'IA?".\\\\
\noindent
Da questa sfida nasce \textbf{(IA)'M FINE}. Il nome è un gioco di parole che richiama la tipica frase "Sto bene" (I'm fine), utilizzata dal giocatore per indicare che non desidera ricevere ulteriori carte.\\\\
\noindent
Per chi avesse bisogno di un ripasso delle regole ufficiali, sono disponibili al seguente indirizzo: \\
\url{https://www.sisal.it/Store/Skill/Download/setteemezzo-ga.pdf}.

\section{Definizione del problema} 
In sintesi, il gioco del Sette e Mezzo è un gioco di carte della tradizione italiana che contrappone un giocatore al banco. Si utilizza un mazzo di 40 carte. Le figure valgono mezzo punto, mentre le carte dall'asso al sette mantengono il loro valore nominale.\\\\
\noindent
Dal punto di vista dell'Intelligenza Artificiale, il gioco rappresenta un problema decisionale in condizioni di incertezza, caratterizzato da informazione parziale e dinamiche stocastiche. La sfida risiede nel calcolare l'azione ottimale ("Carta" o "Sto Bene") basandosi esclusivamente sulle carte visibili e sulla stima di quelle rimanenti.

\subsection{Obiettivi} 
L'obiettivo principale del progetto è sviluppare un agente intelligente in grado di apprendere autonomamente la strategia ottimale per il gioco del Sette e Mezzo. L'agente non deve conoscere a priori le regole probabilistiche del mazzo, ma deve derivarle attraverso l'esperienza (approccio \textit{model-free}).
Nello specifico, gli obiettivi sono:
\begin{itemize}
    \item Implementare un algoritmo di Reinforcement Learning (Q-Learning) che massimizzi il tasso di vittoria contro un "Banco" programmato con regole deterministiche (il banco tira sempre se ha meno di 5).
    \item Gestire la complessità della carta "Matta" (Re di Denari) e delle figure.
    \item Creare un'interfaccia interattiva per visualizzare il processo decisionale dell'IA e permettere a un utente umano di sfidarla.
\end{itemize}

\subsection{Specifica PEAS} 
La descrizione formale dell'ambiente operativo dell'agente è definita secondo il modello PEAS (\textit{Performance, Environment, Actuators, Sensors}):

\begin{itemize}
    \item \textbf{Performance (Misure di Prestazione):}
    \begin{itemize}
        \item +1 punto per la vittoria (punteggio $>$ banco oppure banco sballa).
        \item -1 punto per la sconfitta (sballo, punteggio $\leq$ banco).
        \item L'obiettivo è massimizzare la somma delle ricompense accumulate nel tempo.
    \end{itemize}
    
    \item \textbf{Environment (Ambiente):}
    \begin{itemize}
        \item Tavolo da gioco virtuale.
        \item Mazzo di 40 carte italiane (con rimozione del Re di Denari e inserimento della Matta).
        \item Opponente (il Banco).
    \end{itemize}
    
    \item \textbf{Actuators (Attuatori/Azioni):}
    \begin{itemize}
        \item \textit{Carta} (Hit): Chiedere una nuova carta al mazziere.
        \item \textit{Sto Bene} (Stand): Terminare il turno e confrontare il punteggio.
    \end{itemize}
    
    \item \textbf{Sensors (Sensori/Percezioni):}
    \begin{itemize}
        \item Somma corrente delle proprie carte (inclusa la gestione dinamica del valore della Matta).
        \item Carta visibile del Banco (informazione parziale).
    \end{itemize}
\end{itemize}

\subsubsection{Caratteristiche dell’ambiente} 
L'ambiente di gioco del Sette e Mezzo presenta le seguenti proprietà:
\begin{itemize}
    \item \textbf{Parzialmente Osservabile:} L'agente non vede la carta coperta del banco né le carte rimaste nel mazzo.
    \item \textbf{Stocastico:} L'estrazione della carta successiva è aleatoria.
    \item \textbf{Statico:} L'ambiente non cambia mentre l'agente sta "pensando".
    \item \textbf{Discreto:} Esiste un numero finito di stati e azioni.
    \item \textbf{Agente Singolo:} L'agente gioca contro una logica fissa (la natura/il banco), non contro un altro agente adattivo (nel contesto attuale).
\end{itemize}

\subsection{Analisi del problema} 

\section{Soluzione del problema}

\section{Riferimenti}

\printbibliography

\end{document}