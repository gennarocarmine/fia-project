\documentclass[12pt]{article}

% --- Lingua e Codifica ---
\usepackage[italian,english]{babel}

% --- Matematica ---
\usepackage{amsmath}
\usepackage{mathtools}
\usepackage{amssymb}
\usepackage{bm}

% --- Colori e Grafica ---
\usepackage[table,xcdraw,svgnames]{xcolor} 
\usepackage{graphicx} % Required for inserting images & resizing

% --- Tabelle ---
\usepackage{array}
\usepackage{multirow}
\usepackage{longtable}
\usepackage{tabularx}
\usepackage{booktabs}
\usepackage{adjustbox}

% --- Formattazione e Layout ---
\usepackage[italian]{minitoc}
\usepackage{fancybox}
\usepackage{fancyhdr}
\usepackage{lscape}
\usepackage{placeins}
\usepackage{float}
\usepackage{caption}
\usepackage{soul}

% --- Utilità e Codice ---
\usepackage{verbatim}
\usepackage{url}
\usepackage{listings}
\usepackage{makeidx}
\usepackage{comment}

% --- Bibliografia ---
\usepackage{biblatex}
\addbibresource{sample.bib}

% --- Collegamenti (Caricare per ultimo) ---

%\titleformat{\chapter}{\normalfont\huge}{\textbf\thechapter.}{20pt}%{\huge\textbf}

%inizio documento
\begin{document}
\selectlanguage{italian}

%inizio copertina
\begin{titlepage}
\begin{center}
	\begin{figure}
    	\includegraphics[width=3.0cm, height=3.0cm]{images/unisa.png}
    	\centering
    \end{figure}
	{\Large Università degli Studi di Salerno}\\[0.2truecm]
	{\large Dipartimento di Informatica\\Corso di Laurea Triennale in Informatica}\\
	\hrulefill
	\vfill
	{\large Fondamenti di Intelligenza Artificiale (FIA)}\\[0.2truecm]
    %{\large Project Proposal}\\[0.2truecm]
	%{\Large Informatica}\\
	\vfill
	{\LARGE {\bf (IA)'M FINE: IA PER IL GIOCO DEL SETTE E MEZZO}}
	
	\vfill\vfill
	
	
	{\bf Docente} \hfill {\bf Studenti}\ \hfill  {\bf Matricola}\  \\
	Prof.  Fabio Palomba \hfill Tozza Gennaro Carmine \hfill 0512120382 \\
    \hfill \ \ \ \ \ \ \ \ \ \ \ \ \ \ \ \ Valva Lorenzo \hfill 0512119639 \\
	\vfill
    [gihub]
    \vfill
	\hrulefill 
	\begin{center} Anno Accademico 2025-2026 \end{center}
	
\end{center}
\end{titlepage}
%fine copertina

\section{Introduzione}
Il periodo festivo porta con sé panettoni, calze e gli immancabili parenti che si lamentano dell'Intelligenza Artificiale mentre guardano video di gattini sui social. Ma porta anche le giocate a carte. È stato proprio durante l'ennesima partita persa a Sette e Mezzo che è sorta la domanda: "E se al nostro posto giocasse un'IA?".\\\\
\noindent
Da questa sfida nasce \textbf{(IA)'M FINE}. Il nome deriva dalla tipica frase che si esclama (o si pensa) quando le carte sono favorevoli e decidiamo di "stare".\\\\
\noindent
Per chi avesse bisogno di un ripasso delle regole ufficiali, sono disponibili al seguente indirizzo: \\
\url{https://www.sisal.it/Store/Skill/Download/setteemezzo-ga.pdf}.


\section{Definizione del problema} 

\subsection{Obiettivi} 

\subsection{Specifica PEAS} 

\subsubsection{Caratteristiche dell’ambiente} 

\subsection{Analisi del problema} 

\section{Soluzione del problema}

\section{Riferimenti}

\printbibliography

\end{document}